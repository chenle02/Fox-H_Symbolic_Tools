\documentclass[11pt]{article}

\input{../latex_sources/preamble.tex}
\addbibresource{../latex_sources/Fox-H_biber.bib}
\begin{document}

\section{Example \url{FoxH-Bessel-K_2_9_19.wls}}

\paragraph{File content}

\inputminted{text}{FoxH-Bessel-K_2_9_19.wls}

\paragraph{Fox H-function}

\begin{align*}
  \FoxH
    {2,0}
    {0,2}
    {\cdot}
    {}
    {\left(\frac{a-\eta }{2}, 1\right), \left(\frac{a+\eta }{2}, 1\right)}
\end{align*}

\begin{align*}
  \FoxHext
    {2,0}
    {0,2}
    {\cdot}
    {}
    {}
    {\left(\frac{a-\eta }{2}, 1\right), \left(\frac{a+\eta }{2}, 1\right)}
    {}
\end{align*}

\paragraph{Summary}

\begin{align*}
  a^*    & = 2 \\
  \Delta & = 2 \\
  \delta & = 1 \\
  \mu    & = a-1 \\
  a_1^*  & = 2 \\
  a_2^*  & = 0 \\
  \xi    & = a \\
  c^*    & = 1 \\
\end{align*}

\paragraph{Poles}

\noindent\textbf{1. First eight poles from upper front list}

\begin{align*}
  a_{i,k} = 
  \{\}^T 
\end{align*}
\noindent\textbf{2. First eight poles from lower front list}

\begin{align*}
  b_{j,\ell} = 
  \left(
\begin{array}{cc}
 \frac{\eta -a}{2} & \frac{1}{2} (-a-\eta ) \\
 \frac{1}{2} (-a+\eta -2) & \frac{1}{2} (-a-\eta -2) \\
 \frac{1}{2} (-a+\eta -4) & \frac{1}{2} (-a-\eta -4) \\
 \frac{1}{2} (-a+\eta -6) & \frac{1}{2} (-a-\eta -6) \\
 \frac{1}{2} (-a+\eta -8) & \frac{1}{2} (-a-\eta -8) \\
 \frac{1}{2} (-a+\eta -10) & -\frac{a}{2}-\frac{\eta }{2}-5 \\
 \frac{1}{2} (-a+\eta -12) & -\frac{a}{2}-\frac{\eta }{2}-6 \\
 \frac{1}{2} (-a+\eta -14) & -\frac{a}{2}-\frac{\eta }{2}-7 \\
\end{array}
\right)^T 
\end{align*}

\printbibliography[title={References}]

\end{document}