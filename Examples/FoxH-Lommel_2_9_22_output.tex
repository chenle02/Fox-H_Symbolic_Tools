\documentclass[11pt]{article}

\input{../latex_sources/preamble.tex}
\addbibresource{../latex_sources/Fox-H_biber.bib}
\begin{document}

\section{Example \url{FoxH-Lommel_2_9_22.wls}}

\paragraph{File content}

\inputminted{text}{FoxH-Lommel_2_9_22.wls}

\paragraph{Fox H-function}

\begin{align*}
  \FoxH
    {3,1}
    {1,3}
    {\cdot}
    {\left(\frac{\mu +1}{2}, 1\right)}
    {\left(\frac{\mu +1}{2}, 1\right), \left(\frac{\eta }{2}, 1\right), \left(-\frac{\eta }{2}, 1\right)}
\end{align*}

\begin{align*}
  \FoxHext
    {3,1}
    {1,3}
    {\cdot}
    {\left(\frac{\mu +1}{2}, 1\right)}
    {}
    {\left(\frac{\mu +1}{2}, 1\right), \left(\frac{\eta }{2}, 1\right), \left(-\frac{\eta }{2}, 1\right)}
    {}
\end{align*}

\paragraph{Summary}

\begin{align*}
  a^*    & = 4 \\
  \Delta & = 2 \\
  \delta & = 1 \\
  \mu    & = -1 \\
  a_1^*  & = 3 \\
  a_2^*  & = 1 \\
  \xi    & = \mu +1 \\
  c^*    & = 2 \\
\end{align*}

\paragraph{Poles}

\noindent\textbf{1. First eight poles from upper front list}

\begin{align*}
  a_{i,k} = 
  \left(
\begin{array}{c}
 \frac{1-\mu }{2} \\
 \frac{3-\mu }{2} \\
 \frac{5-\mu }{2} \\
 \frac{7-\mu }{2} \\
 \frac{9-\mu }{2} \\
 \frac{11-\mu }{2} \\
 \frac{13-\mu }{2} \\
 \frac{15-\mu }{2} \\
\end{array}
\right)^T 
\end{align*}
\noindent\textbf{2. First eight poles from lower front list}

\begin{align*}
  b_{j,\ell} = 
  \left(
\begin{array}{ccc}
 \frac{1}{2} (-\mu -1) & -\frac{\eta }{2} & \frac{\eta }{2} \\
 \frac{1}{2} (-\mu -3) & -\frac{\eta }{2}-1 & \frac{\eta -2}{2} \\
 \frac{1}{2} (-\mu -5) & -\frac{\eta }{2}-2 & \frac{\eta -4}{2} \\
 \frac{1}{2} (-\mu -7) & -\frac{\eta }{2}-3 & \frac{\eta -6}{2} \\
 \frac{1}{2} (-\mu -9) & -\frac{\eta }{2}-4 & \frac{\eta -8}{2} \\
 \frac{1}{2} (-\mu -11) & -\frac{\eta }{2}-5 & \frac{\eta }{2}-5 \\
 \frac{1}{2} (-\mu -13) & -\frac{\eta }{2}-6 & \frac{\eta }{2}-6 \\
 \frac{1}{2} (-\mu -15) & -\frac{\eta }{2}-7 & \frac{\eta }{2}-7 \\
\end{array}
\right)^T 
\end{align*}

\printbibliography[title={References}]

\end{document}