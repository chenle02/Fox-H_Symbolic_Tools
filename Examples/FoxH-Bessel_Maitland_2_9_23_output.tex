\documentclass[11pt]{article}

\input{../latex_sources/preamble.tex}
\addbibresource{../latex_sources/Fox-H_biber.bib}
\begin{document}

\section{Example \url{FoxH-Bessel_Maitland_2_9_23.wls}}

\paragraph{File content}

\inputminted{text}{FoxH-Bessel_Maitland_2_9_23.wls}

\paragraph{Fox H-function}

\begin{align*}
  \FoxH
    {1,0}
    {0,2}
    {\cdot}
    {}
    {\left(0, 1\right), \left(-\nu, \mu\right)}
\end{align*}

\begin{align*}
  \FoxHext
    {1,0}
    {0,2}
    {\cdot}
    {}
    {}
    {\left(0, 1\right)}
    {\left(-\nu, \mu\right)}
\end{align*}

\paragraph{Summary}

\begin{align*}
  a^*    & = 1-\mu \\
  \Delta & = \mu +1 \\
  \delta & = \text{ComplexInfinity} \\
  \mu    & = -\nu -1 \\
  a_1^*  & = 1 \\
  a_2^*  & = -\mu \\
  \xi    & = \nu \\
  c^*    & = 0 \\
\end{align*}

\paragraph{Poles}

\noindent\textbf{1. First eight poles from upper front list}

\begin{align*}
  a_{i,k} = 
  \{\}
\end{align*}
\noindent\textbf{2. First eight poles from lower front list}

\begin{align*}
  b_{j,\ell} = 
  \left(
\begin{array}{cccccccc}
 0 & -1 & -2 & -3 & -4 & -5 & -6 & -7 \\
\end{array}
\right)
\end{align*}

\printbibliography[title={References}]

\end{document}