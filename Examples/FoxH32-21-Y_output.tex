\documentclass[11pt]{article}

\input{../latex_sources/preamble.tex}
\addbibresource{../latex_sources/Fox-H_biber.bib}
\begin{document}

\section{Example \url{FoxH32-21-Y.wls}}

\paragraph{File content}

\inputminted{text}{FoxH32-21-Y.wls}

\paragraph{Fox H-function}

\begin{align*}
  \FoxH
    {2,1}
    {2,3}
    {\cdot}
    {\left(1, 1\right), \left(\beta +\gamma, \beta\right)}
    {\left(\frac{d}{2}, \frac{\alpha }{2}\right), \left(1, 1\right), \left(1, \frac{\alpha }{2}\right)}
\end{align*}

\begin{align*}
  \FoxHext
    {2,1}
    {2,3}
    {\cdot}
    {\left(1, 1\right)}
    {\left(\beta +\gamma, \beta\right)}
    {\left(\frac{d}{2}, \frac{\alpha }{2}\right), \left(1, 1\right)}
    {\left(1, \frac{\alpha }{2}\right)}
\end{align*}

\paragraph{Summary}

\begin{align*}
  a^*    & = 2-\beta \\
  \Delta & = \alpha -\beta \\
  \delta & = 2^{-\alpha } \alpha ^{\alpha } \beta ^{-\beta } \\
  \mu    & = \frac{1}{2} (-2 \beta -2 \gamma +d+1) \\
  a_1^*  & = \frac{1}{2} (\alpha -2 \beta +2) \\
  a_2^*  & = 1-\frac{\alpha }{2} \\
  \xi    & = \frac{1}{2} (d-2 (\beta +\gamma -1)) \\
  c^*    & = \frac{1}{2} \\
\end{align*}

\paragraph{Poles}

\noindent\textbf{1. First eight poles from upper front list}

\begin{align*}
  a_{i,k} = 
  \left(
\begin{array}{cccccccc}
 0 & 1 & 2 & 3 & 4 & 5 & 6 & 7 \\
\end{array}
\right)
\end{align*}
\noindent\textbf{2. First eight poles from lower front list}

\begin{align*}
  b_{j,\ell} = 
  \left(
\begin{array}{cccccccc}
 -\frac{d}{\alpha } & -\frac{d+2}{\alpha } & -\frac{d+4}{\alpha } & -\frac{d+6}{\alpha } & -\frac{d+8}{\alpha } & -\frac{d+10}{\alpha } & -\frac{d+12}{\alpha } & -\frac{d+14}{\alpha } \\
 -1 & -2 & -3 & -4 & -5 & -6 & -7 & -8 \\
\end{array}
\right)
\end{align*}

\printbibliography[title={References}]

\end{document}