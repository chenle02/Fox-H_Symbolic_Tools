\documentclass[11pt]{article}


\usepackage{amsmath, url, hyperref, minted, fontspec}
\setmonofont{DejaVu Sans Mono}[Scale=MatchLowercase]
\newcommand{\FoxH}[5]{H_{#2}^{#1}\left(#3\:\middle\vert\: \begin{array}{l}#4\\[0.4em] #5\end{array}\right)}
\newcommand{\FoxHext}[7]{
  \renewcommand{\arraystretch}{1.5} % Adjust the factor 1.5 as needed
  H_{#2}^{#1}\left(#3\:\middle\vert\:
  \begin{array}{c|c}
    #4 & #5 \\ \hline
    #6 & #7
  \end{array}
  \right)
}
\renewcommand{\arraystretch}{1.8}

% -------------------------------------------------
% BibTex Setup
% -------------------------------------------------
\usepackage[strict=true,style=english]{csquotes}
\setlength{\parskip}{0.5cm plus4mm minus3mm}
\usepackage[
  backend=biber,
  style=alphabetic,
  natbib=true,
  abbreviate=true
  ]{biblatex}
\addbibresource{./Examples/biblatex-examples.bib}

\addbibresource{../latex_sources/Fox-H_biber.bib}
\begin{document}

\section{Example \url{FoxH-Macdonald_function_2_9_33.wls}}

\paragraph{File content}

\inputminted{text}{FoxH-Macdonald_function_2_9_33.wls}

\paragraph{Fox H-function}

\begin{align*}
  \FoxH
    {2,1}
    {1,2}
    {\cdot}
    {\left(\gamma -\frac{1}{c}+1, \frac{1}{c}\right)}
    {\left(c \gamma, 1\right), \left(0, \frac{1}{c}\right)}
\end{align*}

\begin{align*}
  \FoxHext
    {2,1}
    {1,2}
    {\cdot}
    {\left(\gamma -\frac{1}{c}+1, \frac{1}{c}\right)}
    {}
    {\left(c \gamma, 1\right), \left(0, \frac{1}{c}\right)}
    {}
\end{align*}

\paragraph{Summary}

\begin{align*}
  a^*    & = \frac{c+2}{c} \\
  \Delta & = 1 \\
  \delta & = \text{Indeterminate} \\
  \mu    & = c \gamma -\gamma +\frac{1}{c}-\frac{3}{2} \\
  a_1^*  & = \frac{1}{c}+1 \\
  a_2^*  & = \frac{1}{c} \\
  \xi    & = c \gamma +\gamma -\frac{1}{c}+1 \\
  c^*    & = \frac{3}{2} \\
\end{align*}

\paragraph{Poles}

\noindent\textbf{1. First eight poles from upper front list}

\begin{align*}
  a_{i,k} = 
  \left(
\begin{array}{cccccccc}
 1-c \gamma  & -\gamma  c+c+1 & 1-c (\gamma -2) & 1-c (\gamma -3) & 1-c (\gamma -4) & 1-c (\gamma -5) & 1-c (\gamma -6) & 1-c (\gamma -7) \\
\end{array}
\right)
\end{align*}
\noindent\textbf{2. First eight poles from lower front list}

\begin{align*}
  b_{j,\ell} = 
  \left(
\begin{array}{cccccccc}
 -c \gamma  & -c \gamma -1 & -c \gamma -2 & -c \gamma -3 & -c \gamma -4 & -c \gamma -5 & -c \gamma -6 & -c \gamma -7 \\
 0 & -c & -2 c & -3 c & -4 c & -5 c & -6 c & -7 c \\
\end{array}
\right)
\end{align*}

\printbibliography[title={References}]

\end{document}