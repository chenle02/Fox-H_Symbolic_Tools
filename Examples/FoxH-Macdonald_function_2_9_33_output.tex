\documentclass[11pt]{article}

\input{../latex_sources/preamble.tex}
\addbibresource{../latex_sources/Fox-H_biber.bib}
\begin{document}

\section{Example \url{FoxH-Macdonald_function_2_9_33.wls}}

\paragraph{File content}

\inputminted{text}{FoxH-Macdonald_function_2_9_33.wls}

\paragraph{Fox H-function}

\begin{align*}
  \FoxH
    {2,0}
    {1,2}
    {\cdot}
    {\left(\gamma -\frac{1}{N}+1, \frac{1}{N}\right)}
    {\left(\gamma  N, 1\right), \left(0, \frac{1}{N}\right)}
\end{align*}

\begin{align*}
  \FoxHext
    {2,0}
    {1,2}
    {\cdot}
    {}
    {\left(\gamma -\frac{1}{N}+1, \frac{1}{N}\right)}
    {\left(\gamma  N, 1\right), \left(0, \frac{1}{N}\right)}
    {}
\end{align*}

\paragraph{Summary}

\begin{align*}
  a^*    & = 1 \\
  \Delta & = 1 \\
  \delta & = \text{Indeterminate} \\
  \mu    & = -\gamma +\gamma  N+\frac{1}{N}-\frac{3}{2} \\
  a_1^*  & = 1 \\
  a_2^*  & = 0 \\
  \xi    & = -\gamma +\gamma  N+\frac{1}{N}-1 \\
  c^*    & = \frac{1}{2} \\
\end{align*}

\paragraph{Poles}

\noindent\textbf{1. First eight poles from upper front list}

\begin{align*}
  a_{i,k} = 
  \{\}^T 
\end{align*}
\noindent\textbf{2. First eight poles from lower front list}

\begin{align*}
  b_{j,\ell} = 
  \left(
\begin{array}{cc}
 \gamma  (-N) & 0 \\
 \gamma  (-N)-1 & -N \\
 \gamma  (-N)-2 & -2 N \\
 \gamma  (-N)-3 & -3 N \\
 \gamma  (-N)-4 & -4 N \\
 \gamma  (-N)-5 & -5 N \\
 \gamma  (-N)-6 & -6 N \\
 \gamma  (-N)-7 & -7 N \\
\end{array}
\right)^T 
\end{align*}

\printbibliography[title={References}]

\end{document}