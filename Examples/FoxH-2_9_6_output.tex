\documentclass[11pt]{article}

\input{../latex_sources/preamble.tex}
\addbibresource{../latex_sources/Fox-H_biber.bib}
\begin{document}

\section{Example \url{FoxH-2_9_6.wls}}

\paragraph{File content}

\inputminted{text}{FoxH-2_9_6.wls}

\paragraph{Fox H-function}

\begin{align*}
  \FoxH
    {1,0}
    {1,1}
    {\cdot}
    {\left(\alpha +\beta +1, 1\right)}
    {\left(\alpha, 1\right)}
\end{align*}

\begin{align*}
  \FoxHext
    {1,0}
    {1,1}
    {\cdot}
    {}
    {\left(\alpha +\beta +1, 1\right)}
    {\left(\alpha, 1\right)}
    {}
\end{align*}

\paragraph{Summary}

\begin{align*}
  a^*    & = 0 \\
  \Delta & = 0 \\
  \delta & = 1 \\
  \mu    & = -\beta -1 \\
  a_1^*  & = 0 \\
  a_2^*  & = 0 \\
  \xi    & = -\beta -1 \\
  c^*    & = 0 \\
\end{align*}

\paragraph{Poles}

\noindent\textbf{1. First eight poles from upper front list}

\begin{align*}
  a_{i,k} = 
  \{\}
\end{align*}
\noindent\textbf{2. First eight poles from lower front list}

\begin{align*}
  b_{j,\ell} = 
  \left(
\begin{array}{cccccccc}
 -\alpha  & -\alpha -1 & -\alpha -2 & -\alpha -3 & -\alpha -4 & -\alpha -5 & -\alpha -6 & -\alpha -7 \\
\end{array}
\right)
\end{align*}

\printbibliography[title={References}]

\end{document}