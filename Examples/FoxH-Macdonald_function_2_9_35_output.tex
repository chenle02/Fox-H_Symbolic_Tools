\documentclass[11pt]{article}

\input{../latex_sources/preamble.tex}
\addbibresource{../latex_sources/Fox-H_biber.bib}
\begin{document}

\section{Example \url{FoxH-Macdonald_function_2_9_35.wls}}

\paragraph{File content}

\inputminted{text}{FoxH-Macdonald_function_2_9_35.wls}

\paragraph{Fox H-function}

\begin{align*}
  \FoxH
    {2,0}
    {1,2}
    {\cdot}
    {\left(1-\frac{\sigma +1}{\beta }, \frac{1}{\beta }\right)}
    {\left(0, 1\right), \left(-\frac{\sigma }{\beta }-\gamma, \frac{1}{\beta }\right)}
\end{align*}

\begin{align*}
  \FoxHext
    {2,0}
    {1,2}
    {\cdot}
    {}
    {\left(1-\frac{\sigma +1}{\beta }, \frac{1}{\beta }\right)}
    {\left(0, 1\right), \left(-\frac{\sigma }{\beta }-\gamma, \frac{1}{\beta }\right)}
    {}
\end{align*}

\paragraph{Summary}

\begin{align*}
  a^*    & = 1 \\
  \Delta & = 1 \\
  \delta & = \text{Indeterminate} \\
  \mu    & = \frac{1}{\beta }-\gamma -\frac{3}{2} \\
  a_1^*  & = 1 \\
  a_2^*  & = 0 \\
  \xi    & = \frac{1}{\beta }-\gamma -1 \\
  c^*    & = \frac{1}{2} \\
\end{align*}

\paragraph{Poles}

\noindent\textbf{1. First eight poles from upper front list}

\begin{align*}
  a_{i,k} = 
  \{\}^T 
\end{align*}
\noindent\textbf{2. First eight poles from lower front list}

\begin{align*}
  b_{j,\ell} = 
  \left(
\begin{array}{cc}
 0 & \beta  \gamma +\sigma  \\
 -1 & \beta  (\gamma -1)+\sigma  \\
 -2 & \beta  (\gamma -2)+\sigma  \\
 -3 & \beta  (\gamma -3)+\sigma  \\
 -4 & \beta  (\gamma -4)+\sigma  \\
 -5 & \beta  (\gamma -5)+\sigma  \\
 -6 & \beta  (\gamma -6)+\sigma  \\
 -7 & \beta  (\gamma -7)+\sigma  \\
\end{array}
\right)^T 
\end{align*}

\printbibliography[title={References}]

\end{document}