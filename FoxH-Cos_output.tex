\documentclass{article}
\usepackage{amsmath}
\newcommand{\FoxH}[5]{H_{#2}^{#1}\left(#3\:\middle\vert\: \begin{subarray}{l}#4\\[0.4em] #5\end{subarray}\right)}
\begin{document}
\begin{align*}
\FoxH{2,0}{0,1}{\cdot}{\left(1, \frac{1}{2}\right), \left(\frac{1}{2}, \frac{1}{2}\right)}{}
\end{align*}
\noindent\textbf{Summary}
\begin{align*}
a^* &= 0 \\
\Delta &= -1 \\
\delta &= 0 \\
\mu &= -\frac{1}{2} \\
a_1^* &= -\frac{1}{2} \\
a_2^* &= \frac{1}{2} \\
\xi &= \frac{1}{2} \\
c^* &= 0 \\
\end{align*}
\noindent\textbf{Poles}\\
\noindent\textbf{1. First ten poles from upper front list}
\begin{align*}
a_{i,k} &= \left(
\begin{array}{c}
 0 \\
 2 \\
 4 \\
 6 \\
 8 \\
 10 \\
 12 \\
 14 \\
 16 \\
 18 \\
 20 \\
\end{array}
\right)
\end{align*}
\noindent\textbf{2. First ten poles from lower front list}
\begin{align*}
b_{j,\ell} &= \{\{\},\{\},\{\},\{\},\{\},\{\},\{\},\{\},\{\},\{\},\{\}\}
\end{align*}
\end{document}