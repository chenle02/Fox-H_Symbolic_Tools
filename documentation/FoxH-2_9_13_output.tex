\documentclass[preview]{standalone}

\input{../latex_sources/preamble.tex}
\addbibresource{../latex_sources/Fox-H_biber.bib}
\begin{document}

\subsection{Example \url{FoxH-2_9_13.wls}}

\paragraph{File content}

\inputminted{text}{../Examples/FoxH-2_9_13.wls}

\paragraph{Fox H-function}

\begin{align*}
  \FoxH
    {1,2}
    {2,2}
    {\cdot}
    {\left(\frac{1}{2}, 1\right), \left(1, 1\right)}
    {\left(\frac{1}{2}, 1\right), \left(0, 1\right)}
\end{align*}

\begin{align*}
  \FoxHext
    {1,2}
    {2,2}
    {\cdot}
    {\left(\frac{1}{2}, 1\right), \left(1, 1\right)}
    {}
    {\left(\frac{1}{2}, 1\right)}
    {\left(0, 1\right)}
\end{align*}

\paragraph{Summary}

\begin{align*}
  a^*    & = 2 \\
  \Delta & = 0 \\
  \delta & = 1 \\
  \mu    & = -1 \\
  a_1^*  & = 1 \\
  a_2^*  & = 1 \\
  \xi    & = 2 \\
  c^*    & = 1 \\
\end{align*}

\paragraph{Poles}

\noindent\textbf{1. First eight poles from upper front list}

\begin{align*}
  a_{i,k} = 
  \left(
\begin{array}{cccccccc}
 \frac{1}{2} & \frac{3}{2} & \frac{5}{2} & \frac{7}{2} & \frac{9}{2} & \frac{11}{2} & \frac{13}{2} & \frac{15}{2} \\
 0 & 1 & 2 & 3 & 4 & 5 & 6 & 7 \\
\end{array}
\right)
\end{align*}
\noindent\textbf{2. First eight poles from lower front list}

\begin{align*}
  b_{j,\ell} = 
  \left(
\begin{array}{cccccccc}
 -\frac{1}{2} & -\frac{3}{2} & -\frac{5}{2} & -\frac{7}{2} & -\frac{9}{2} & -\frac{11}{2} & -\frac{13}{2} & -\frac{15}{2} \\
\end{array}
\right)
\end{align*}


\end{document}