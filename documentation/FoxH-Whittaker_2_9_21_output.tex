\documentclass[preview]{standalone}

\input{../latex_sources/preamble.tex}
\addbibresource{../latex_sources/Fox-H_biber.bib}
\begin{document}

\subsection{Example \url{FoxH-Whittaker_2_9_21.wls}}

\paragraph{File content}

\inputminted{text}{../Examples/FoxH-Whittaker_2_9_21.wls}

\paragraph{Fox H-function}

\begin{align*}
  \FoxH
    {2,0}
    {1,2}
    {\cdot}
    {\left(a-\lambda +1, 1\right)}
    {\left(a+\mu +\frac{1}{2}, 1\right), \left(a-\mu +\frac{1}{2}, 1\right)}
\end{align*}

\begin{align*}
  \FoxHext
    {2,0}
    {1,2}
    {\cdot}
    {}
    {\left(a-\lambda +1, 1\right)}
    {\left(a+\mu +\frac{1}{2}, 1\right), \left(a-\mu +\frac{1}{2}, 1\right)}
    {}
\end{align*}

\paragraph{Summary}

\begin{align*}
  a^*    & = 1 \\
  \Delta & = 1 \\
  \delta & = 1 \\
  \mu    & = a+\lambda -\frac{1}{2} \\
  a_1^*  & = 1 \\
  a_2^*  & = 0 \\
  \xi    & = a+\lambda \\
  c^*    & = \frac{1}{2} \\
\end{align*}

\paragraph{Poles}

\noindent\textbf{1. First eight poles from upper front list}

\begin{align*}
  a_{i,k} = 
  \{\}^T 
\end{align*}
\noindent\textbf{2. First eight poles from lower front list}

\begin{align*}
  b_{j,\ell} = 
  \left(
\begin{array}{cc}
 -a-\mu -\frac{1}{2} & -a+\mu -\frac{1}{2} \\
 -a-\mu -\frac{3}{2} & -a+\mu -\frac{3}{2} \\
 -a-\mu -\frac{5}{2} & -a+\mu -\frac{5}{2} \\
 -a-\mu -\frac{7}{2} & -a+\mu -\frac{7}{2} \\
 -a-\mu -\frac{9}{2} & -a+\mu -\frac{9}{2} \\
 -a-\mu -\frac{11}{2} & -a+\mu -\frac{11}{2} \\
 -a-\mu -\frac{13}{2} & -a+\mu -\frac{13}{2} \\
 -a-\mu -\frac{15}{2} & -a+\mu -\frac{15}{2} \\
\end{array}
\right)^T 
\end{align*}


\end{document}