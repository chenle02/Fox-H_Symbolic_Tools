\documentclass{article}
\usepackage{amsmath, url, hyperref, minted}
\newcommand{\FoxH}[5]{H_{#2}^{#1}\left(#3\:\middle\vert\: \begin{array}{l}#4\\[0.4em] #5\end{array}\right)}
\newcommand{\FoxHext}[7]{
  \renewcommand{\arraystretch}{1.5} % Adjust the factor 1.5 as needed
  H_{#2}^{#1}\left(#3\:\middle\vert\:
  \begin{array}{c|c}
    #4 & #5 \\ \hline
    #6 & #7
  \end{array}
  \right)
}
\renewcommand{\arraystretch}{1.8}

\begin{document}

\section{Example \url{test.wls}}

\paragraph{File content}

\begin{minted}{text}
{
  (* Upper List *) {
    (* Upper Front list *) {{1, \[Alpha]^(-1)}, {1, 1}},
    (* Upper Rear List *)  {{Ceil[\[Beta]], \[Beta]}, {1, 1}}
  },
  (* Lower List *) {
    (* Lower Front List *) {{1/2, \[Alpha]/2}, {1, 1}, {3, 3}, {2, 2}},
    (* Lower Rear List *)  {{1, \[Alpha]/2}}
  }
}
\end{minted}

\paragraph{Fox H-function}

\begin{align*}
  \FoxH
    {4,2}
    {4,5}
    {\cdot}
    {\left(1, \frac{1}{\alpha }\right), \left(1, 1\right), \left(\text{Ceil}(\beta ), \beta\right), \left(1, 1\right)}
    {\left(\frac{1}{2}, \frac{\alpha }{2}\right), \left(1, 1\right), \left(3, 3\right), \left(2, 2\right), \left(1, \frac{\alpha }{2}\right)}
\end{align*}

\begin{align*}
  \FoxHext
    {4,2}
    {4,5}
    {\cdot}
    {\left(1, \frac{1}{\alpha }\right), \left(1, 1\right)}
    {\left(\text{Ceil}(\beta ), \beta\right), \left(1, 1\right)}
    {\left(\frac{1}{2}, \frac{\alpha }{2}\right), \left(1, 1\right), \left(3, 3\right), \left(2, 2\right)}
    {\left(1, \frac{\alpha }{2}\right)}
\end{align*}

\paragraph{Summary}

\begin{align*}
  a^*    & = \frac{1}{\alpha }-\beta +6 \\
  \Delta & = \alpha -\frac{1}{\alpha }-\beta +4 \\
  \delta & = \frac{2^{-\alpha } \left(2^{\frac{\alpha }{2}+5} \alpha ^{\alpha /2}+\alpha ^{\alpha }\right)}{\left(\left(\frac{1}{\alpha }\right)^{\frac{1}{\alpha }}+1\right) \left(\beta ^{\beta }+1\right)} \\
  \mu    & = 4-\text{Ceil}(\beta ) \\
  a_1^*  & = \frac{\alpha }{2}-\beta +5 \\
  a_2^*  & = -\frac{\alpha }{2}+\frac{1}{\alpha }+1 \\
  \xi    & = \frac{13}{2}-\text{Ceil}(\beta ) \\
  c^*    & = \frac{3}{2} \\
\end{align*}

\paragraph{Poles}

\noindent\textbf{1. First ten poles from upper front list}

\begin{align*}
  a_{i,k} = 
  \left(
\begin{array}{cc}
 0 & 0 \\
 \alpha  & 1 \\
 2 \alpha  & 2 \\
 3 \alpha  & 3 \\
 4 \alpha  & 4 \\
 5 \alpha  & 5 \\
 6 \alpha  & 6 \\
 7 \alpha  & 7 \\
 8 \alpha  & 8 \\
 9 \alpha  & 9 \\
 10 \alpha  & 10 \\
\end{array}
\right)
\end{align*}
\noindent\textbf{2. First ten poles from lower front list}

\begin{align*}
  b_{j,\ell} = 
  \left(
\begin{array}{cccc}
 -\frac{1}{\alpha } & -1 & -1 & -1 \\
 -\frac{3}{\alpha } & -2 & -\frac{4}{3} & -\frac{3}{2} \\
 -\frac{5}{\alpha } & -3 & -\frac{5}{3} & -2 \\
 -\frac{7}{\alpha } & -4 & -2 & -\frac{5}{2} \\
 -\frac{9}{\alpha } & -5 & -\frac{7}{3} & -3 \\
 -\frac{11}{\alpha } & -6 & -\frac{8}{3} & -\frac{7}{2} \\
 -\frac{13}{\alpha } & -7 & -3 & -4 \\
 -\frac{15}{\alpha } & -8 & -\frac{10}{3} & -\frac{9}{2} \\
 -\frac{17}{\alpha } & -9 & -\frac{11}{3} & -5 \\
 -\frac{19}{\alpha } & -10 & -4 & -\frac{11}{2} \\
 -\frac{21}{\alpha } & -11 & -\frac{13}{3} & -6 \\
\end{array}
\right)
\end{align*}

\end{document}